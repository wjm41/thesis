%!TEX root = ../thesis.tex
%*******************************************************************************
%****************************** Third Chapter **********************************
%*******************************************************************************
\chapter{Future Work}


\section{Short-Term: Continutation of ongoing work}
Given the relative success of work so far, the immediate plan is to continue ongoing research to completion. After developing a more appropriate benchmark dataset for evaluating the performance of reaction prediction models, the intention is to write up the results of chapter \ref{chap:MolTrans} within the next month, potentially following up with an attempt to `fix' the model which may also be a contribution to the field of NLP in particular regarding the modelling of long sequences. The investigation into SOAP descriptors for QSAR should hopefully be concluded shortly after resubmitting the results to JMedChem. If the drug candidates proposed by the Siamese GNN prove potent, then there is a strong incentive to refine and retrospectively validate the model on historical data and publish the methodology.

In addition, the COVID Moonshot project will likely continue for another $\sim$8 months and I will continue my participation of the project given the obvious urgency of the pandemic. No doubt this enterprise will remain a fruitful source of interesting problems with real experimental data, which will hopefully lead to innovative solutions. Research will probably be on continuing optimisation of existing molecular series, or searching for alternative backup series while the most promising series undergo \textit{in vivo} toxicity screening.

\section{Long-Term: Investigation of new modalities}
Although using artificial intelligence for optimising the small-molecule drug discovery process is undoubtedly a difficult task, there has been extensive interest from both academia and private industry with many breakthroughs having been made already. The field is maturing to the extent that ML algorithms are already beginning to become part of the commercial design-make-test workflow, such that the remaining challenges are arguably merely an engineering problem. 

The therapeutic space beyond small-molecules, however, is relatively unexplored territory for data-driven techniques. Applying machine learning to this area will likely present even more complex challenges, but the potential impact of developing new modalities far outweigh that of `just' improving small-molecule QSAR modelling. While the potential areas of research are numerous, thus far two topics of interest have been identified:

\begin{itemize}
    \item functionalisation of flexible biomolecules (glycans, peptides),
    \item understand/design self-assembled nanostructures for drug delivery.
\end{itemize}

Both of these topics involve structures that are larger and less well-understood by medicinal (bio)chemists because of their energetic/entropic complexity. This is a promising area where I could combine physics-based intuition for modelling interactions, as well as pragmatic ML for designing models that would be useful in a drug discovery setting.

While no ML in involved in the Test of design-make-test it is nonetheless vital to retain this part of the cycle for proper validation of ML drug discovery methods, for taking the important step from mere concept to real-life data-driven-drugs. Therefore there is an expectation that some form of experimental work will be carried out, likely alongside more experienced collaborators, in the latter stages (2\textsuperscript{nd}-3\textsuperscript{rd} year) of the PhD irrespective of the ultimate direction of research.