% ************************** Thesis Abstract *****************************
% Use `abstract' as an option in the document class to print only the titlepage and the abstract.
\begin{abstract}
    Drug discovery follows a design-make-test cycle of proposing drug compounds, synthesising them, and measuring their bioactivity, which informs the next cycle of compound designs. The challenges associated with each step lead to the long timeline of preclinical pharmaceutical development. This thesis focuses on how we can use machine learning tools to accelerate the design-make-test cycle for faster drug discovery.
    
    We begin with the design of new compounds, looking at the initial stage of fragment-based hit finding where only the 3D coordinates of fragment-protein complexes are available. The standard approach is to “grow” or “merge” nearby fragments based on their binding modes, but fragments typically have low affinity so the road to potency is often long and fraught with false starts. Instead, we can reframe fragment-based hit discovery as a denoising problem - identifying significant pharmacophore distributions from an “ensemble” of fragments amid noise due to weak binders - and employ an unsupervised machine learning method to tackle this problem. We construct a model that screens potential molecules by evaluating whether they recapitulate those fragment-derived pharmacophore distributions. We show that this approach outperforms docking in distinguishing active compounds from inactive ones on historical data. Further, we prospectively find novel hits for SARS-CoV-2 Mpro and the Mac1 domain of SARS-CoV-2 non-structural protein 3 by screening a library of 1 billion molecules.
    
    After identifying hit compounds, we enter the hit-to-lead stage where we wish to optimise their molecular structures to improve bioactivity. Framing bioactivity modelling as active/inactive classification would not allow us to rank compounds based on predicted bioactivity improvement, while the low number of active compounds and the measurement noise make a regression approach challenging. We overcome this challenge with a learning-to-rank framework via a classifier that predicts whether a compound is more or less active than another using the difference in molecular descriptors between the molecules as input. This allows us to make use of inactive data, and threshold the bioactivity differences above measurement noise. Validation on retrospective data for Mpro shows that we can outperform docking on ranking ligands, and we prospectively screen a library of 8.8M molecules and arrive at a potent compound with a novel scaffold.
    
    Throughout the entire course of drug discovery, one needs to find a synthesis route to actually make the molecule. An exciting approach is to use deep learning models trained on patent reaction databases, but they suffer from being opaque black boxes. It is neither clear if the models are making correct predictions because they inferred the salient chemistry, nor is it clear which training data they are relying on to reach a prediction. To address this issue, we developed a workflow for quantitatively interpreting a state-of-the-art deep learning model for reaction prediction. By analysing chemically selective reactions, we show examples of correct reasoning by the model, explain counterintuitive predictions, and identify Clever Hans predictions where the correct answer is reached for the wrong reason due to dataset bias.
    
    Testing a drug candidate typically involves obtaining a pure sample of the molecule, and then measuring its bioactivity in solution via an assay. While necessary for maximum accuracy, compound purification can be time-consuming and costly. We investigated whether we needed compound purification at all for training machine learning bioactivity models by assaying crude reaction mixtures instead of pure samples. This approach allowed us to obtain bioactivity data in higher throughput and train useful models for the identification of false negative assay measurements, as well as prospective screens.
    
    The research presented in this thesis highlights the promise of applying machine learning in accelerating the design-make-test cycle of drug discovery. This thesis concludes by outlining promising research directions for applying machine learning within drug discovery.
\end{abstract}
    
    