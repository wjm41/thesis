%!TEX root = ../thesis.tex
% ******************************* Thesis Appendix B ********************************

\chapter{Experimental Details} \label{appendix:experiments}

\section{Mpro assay}
\section{Mac1 assay}
\section{Crystallographic screening on SARS-CoV-2 nsp3-Mac1}
Crystallographic screening of compounds was performed using Mac1 crystals grown in the P43 space group, following the previously described protocol (PMID: 33853786). Compounds synthesized by Enamine/WuXi were prepared in DMSO to 100 mM and were added to crystallization drops using an Echo 650 liquid handler (Labcyte) (PMID: 28291760). Crystals were soaked at either 10 or 20 mM for 2-4.5 hours, before being vitrified in liquid nitrogen using a Nanuq cryocooling device (Mitegen). Soak times and concentrations are listed in Table S1. Diffraction data were collected at beamlines 12-1 and 12-2 of the Stanford Synchrotron Radiation Lightsource (SSRL). The data collection strategy and statistics are listed in Table S1. Compound binding was detected using the PanDDA algorithm (PMID: 28436492) as described previously (PMID: 35794891). PanDDA was initially run using a background map calculated with 34 datasets collected from crystals soaked only in DMSO (annotated as dmso\_34 in Table S1). PanDDA was rerun with a background map calculated using two sets of 35 datasets where no compound binding was detected (annotated as either ssrl\_1 or ssrl\_2 in Table S1). This procedure led to the identification of an additional nine hits (Table S1). 

Compounds were modeled into PanDDA event maps using COOT (PMID: 20383002) with coordinates and restraints generated by phenix.elbow from SMILES strings (PMID: 19770504). Duplicate soaks were performed for most compounds: where the same compound was identified in multiple datasets, the highest occupancy compound was modeled. Both the compound-bound and compound-free coordinates were refined together as a multi-state model following the protocol described previously (PMID: 28436492). Compound occupancy was set based on the background density correction (BDC) value (PMID: 28436492). Refinement statistics are presented in Table S1. Coordinates and structure factor amplitudes have been deposited in the protein data bank (PDB) with the group deposition code G\_1002254. PanDDA input and output files have been uploaded to Zenodo (DOI: 10.5281/zenodo.7231822), and the raw diffraction images are available at https://proteindiffraction.org/.